\PassOptionsToPackage{pdfpagelabels=false}{hyperref} 
\documentclass[12pt,a4paper,sans]{moderncv}

%% ModernCV themes
\moderncvstyle{casual}
\moderncvcolor{blue}
\renewcommand{\familydefault}{\sfdefault}
%\nopagenumbers{}

%\newboolean{articletitles}
%\setboolean{articletitles}{true} % False removes titles in references
%\newboolean{inbibliography}
%\setboolean{inbibliography}{false} %True once you enter the bibliography
%\newboolean{uprightparticles}
%\setboolean{uprightparticles}{false} %True for upright particle symbols
%\addcontentsline{toc}{section}{References}
%\setboolean{inbibliography}{true}
%% Character encoding
\usepackage[utf8]{inputenc}
\usepackage[english]{babel}
%% Adjust the page margins
\usepackage[scale=0.8]{geometry}
\usepackage{amsmath}
%\usepackage[utf8]{inputenc}
\usepackage[T1]{fontenc}
\usepackage{graphicx}
%\usepackage{cite} % Allows for ranges in citations
\usepackage{mciteplus}
\usepackage[numbers]{natbib}
%\usepackage{hyperref}    % Hyperlinks in references
%\usepackage[all]{hypcap} % Internal hyperlinks to floats.
%\usepackage[options]{natbib}
\newboolean{articletitles}
\setboolean{articletitles}{true} % False removes titles in references

\newboolean{uprightparticles}
\setboolean{uprightparticles}{false} %True for upright particle symbols

\newboolean{inbibliography}
\setboolean{inbibliography}{false} %True once you enter the bibliography

\DeclareSymbolFont{extraup}{U}{zavm}{m}{n}
\DeclareMathSymbol{\varheart}{\mathalpha}{extraup}{86}
\DeclareMathSymbol{\vardiamond}{\mathalpha}{extraup}{87}

%\input{lhcb-symbols-def} % Add in the predefined LHCb symbols
%% Personal data
\firstname{Sascha}
\familyname{Stahl}
%\title{Resumé title (optional)}
\address{CERN}{1211 Geneve}
%\mobile{+1~(234)~567~890}
%\phone{+2~(345)~678~901}
%\fax{+3~(456)~789~012}
%\email{john@doe.org}
%\homepage{www.johndoe.com}
%\extrainfo{additional information}
%\photo[64pt][0.4pt]{picture}
%\quote{Some quote (optional)}

%%------------------------------------------------------------------------------
%% Content
%%------------------------------------------------------------------------------
\begin{document}
\makecvtitle
\section{Personal}
\cventry{Birth}{25th February 1985}{Bad Kreuznach, Germany}{}{}{}
\cventry{Nationality}{German}{}{}{}{}
\section{Employment history}
\cventry{11/2014--}{Marie Curie COFUND CERN fellow}{CERN}{}{}{}
\subsection{Responsibilities}
\cventry{01/2016--}{Convener of the subgroup ``CPV, mixing and $b$-hadron properties'' within the Semileptonic working group of LHCb}{the Semileptonic working group at LHCb studies $b$-hadron decays with one lepton in the final state}{}{}{}
\cventry{04/2016--}{Deputy project leader of the LHCb Software trigger project}{the LHCb software trigger is a vital part of the LHCb data acquisition and has continuously extended the physics reach of LHCb}{}{}{}
\cventry{11/2014--04/2016}{Responsible for the reconstruction of the LHCb software trigger}{during which time I tuned and commissioned the software trigger reconstruction to be equivalent to the offline-quality reconstruction for the first time.}{}{}{}

\subsection{Awards}
\cventry{09/2016}{LHCb Early Career Scientist award}{``For having implemented and commissioned the revolutionary changes to the LHC Run-2 high-level-trigger (HLT), including the first widespread deployment of real-time analysis techniques in High Energy Physics. ...''}
{\href{http://lhcb.web.cern.ch/lhcb/Collaboration_prizes/Early_career_awards.html}{LHCb Collaboration prizes}}{}{}  % arguments 3 to 6 can be left empty
\section{Studies and PhD}
\cventry{09/2010-- 07/2014}{PhD in physics}{Ruprecht-Karls Universität Heidelberg}{Heidelberg}{grade: summa cum laude (with highest distinction)}{}  % arguments 3 to 6 can be left empty
\cventry{07/2014}{PhD thesis}{}{}{under the supervision of Prof. Dr. Stephanie Hansmann-Menzemer in the LHCb group of the Physikalisches Institut
in Heidelberg}{Title: ``Measurement of $C\!P$ asymmetry in muon-tagged $D^0 \rightarrow K^-K^+$ and $D^0 \rightarrow \pi^- \pi^+$ decays at LHCb''.}  
\cventry{04/2005--06/2010}{Diploma in physics}{Ruprecht-Karls Universität Heidelberg}{Heidelberg, Germany}{Grade: sehr gut (very good)}{}  % arguments 3 to 6 can be left empty
\cventry{05/2010}{Diploma thesis}{}{}{under the supervision of Prof. Dr. Stephanie Hansmann-Menzemer in the LHCb group of the Physikalisches Institut
in Heidelberg}{Title: ``Reconstruction of displaced tracks and measurement of $K_s$ production rate in 
    proton-proton collisions at $\sqrt{s}=900~\text{GeV}$ at the LHCb experiment''.}  

\section{Talks}
\cventry{07/2015}{EPS-HEP 2015}{Vienna, Austria}{``The LHCb Higher Level Trigger in Run II''}{}{}  % arguments 3 to 6 can be left empty
\cventry{01/2015}{Particle Physics Seminar}{Bonn, Germany}{``Measurements of $C\!P$ asymmetries in charm decays at LHCb''}{}{}  % arguments 3 to 6 can be left empty
\cventry{11/2014}{Graduation ceremony}{Selected speaker by the Combined faculties for the Natural Sciences and Mathematics of the Heidelberg University}{``Teilchen und Antiteilchen, warum und wie messe ich den Unterschied?''}{}{}  % arguments 3 to 6 can be left empty
\cventry{03/2013}{Rencontres de Moriond QCD 2013}{}{``Properties of b and c hadrons at LHCb''}{}{}  % arguments 3 to 6 can be left empty
\cventry{03/2010}{Rencontres de Moriond EW 2010}{Talk in the Young Scientists Forum}{``Tracking performance in V0 reconstruction with first data at LHCb''}{}{}  % arguments 3 to 6 can be left empty

\subsection{Scholarships}
\cventry{09/2010-- 02/2013}{International Max Planck Research School for Precision Tests of Fundamental Symmetries in Particle Physics, 
  Nuclear Physics, Atomic Physics and Astroparticle Physics at the University of Heidelberg}
{Max-Planck--Institut für Kernphysik}{Heidelberg}{}{}  % arguments 3 to 6 can be left empty

\section{Teaching and Supervision}
During my undergraduate and graduate studies I tutored courses in experimental and theoretical physics.\\
I have supervised 3 bachelor and summer students, 1 master student and closely followed 2 PhD students
working on a variety of topics. \\
In my role as HLT deputy project leader I coordinate the work of a core team of around 5 post-docs. 
%\cventry{11/2015-- 07/2016}{Supervision of Guillaume Falmagne visiting student from ENS Cachan} {``Measurement of $\Lambda_b$ production asymmetry in pp collisions''}{}{}{}{}
%\cventry{04/2015-- 04/2016}{Supervision of Marian Stahl PhD student from Heidelberg University} {``Forward Tracking Improvements for Run2 Data Taking''}{}{}{}{}
%\cventry{07/2015-- 09/2015}{Supervision of Renata Kopecna within the CERN Summer student programme} {``Two-particle correlations in pp collisions''\cite{Kopecna:2051967}}{}{}{}
%\cventry{2012-- 2014}{Following Andreas Jäger, PhD student} {``Measurement of indirect $CP$ asymmetries in $D^0\rightarrow K^-K^+$ and $D^0\rightarrow \pi^-\pi^+$ decays using semileptonic $B$ decays''\cite{Aaij:2015yda}}{}{}{}
%\cventry{2010-- 2014}{Supervision of Bachelor and Summer Students}{}{}{}{}
%\cventry{09/2013-- 02/2014}{Tutor for master course}{Experimental Particle Physics}{}{}{}  % arguments 3 to 6 can be left empty
%\cventry{2008-2011}{Tutor for undergraduate courses}{Theoretical Mechanics, Theoretical Electrodynamics and Theoretical Quantum Dynamics}{}{}{}  % arguments 3 to 6 can be left empty


\newpage

\section{Research activities}
I started my career in high energy physics as a master student in 2009 at a very exciting time
when the LHCb experiment was about to start taking data. 
During my masters thesis I improved a pattern recognition algorithm dedicated to the 
reconstruction of very displaced tracks. I used the algorithm to reconstruct 
the first $K_S^0$ signals and was among the main authors of the first LHCb publication~\cite{LHCb-PAPER-2010-001}.
My contribution was rewarded by the collaboration with a talk in the Young Scientists Forum at Rencontres de Moriond EW in 2010.  
I also contributed to the first LHCb measurement of prompt charm production at $\sqrt{s}=7\;$TeV in the forward region~\cite{LHCb-PAPER-2012-041},
being responsible for the evaluation of detection efficiencies and data-simulation discrepancies.
These were particularly important since a cross-section measurement requires the knowledge of 
absolute, not relative, efficiencies.

I started my PhD in September 2010. The first $B$-physics analyses 
were the measurements of $B$-meson mixing and $C\!P$ violation in mixing. 
Together with one post-doc and two other PhD students from the Heidelberg group,
we measured the mixing frequencies $\Delta m_d$ and $\Delta m_s$, 
the latter with the highest precision at this time~\cite{LHCb-PAPER-2013-006}. Afterwards, I contributed to the measurement 
of the mixing phase $\phi_s$. I studied the decay-time acceptance and the decay-time resolution 
which were crucial inputs to the multi-dimensional fit in the decay-time and angular observables~\cite{LHCb-PAPER-2011-021}. 

To work full speed on my thesis topic I moved to CERN in 2012.
In November 2011 the LHCb experiment had reported evidence of direct $C\!P$ violation in the decay of
charm mesons in the modes $D^0 \rightarrow K^-K^+$ and $D^0 \rightarrow \pi^-\pi^+$. This came as 
a surprise as it deviated from the Standard Model predictions.
Together with two colleagues we came up with the novel idea to cross-check the result on an independent 
data sample, namely exploiting $D$~mesons from semileptonic $B$~decays.
I~developed independent measurements for detection and 
reconstruction asymmetries which was the main challenge of the $C\!P$ asymmetry measurement.  
The analysis, which did not confirm the original signal, was followed with a 
lot of interest by the collaboration and was one of the most scrutinized LHCb analyses.
I was the contact person for the internal review process
and the analysis was published in~\cite{LHCb-PAPER-2013-003,LHCb-PAPER-2014-013}.
In parallel to my analysis activities I joined the High Level Trigger team and was one of the core 
people charged with ensuring the smooth running of the trigger at the restart of the data taking in spring 2012. 
I profited from my background in track reconstruction, tuning reconstruction and track quality 
criteria in order to optimize the efficiency of physics signals given the computing requirements.

In 2013 the preparation of the LHCb upgrade entered an important phase.
I was asked to contribute to a task force to setup the reconstruction sequence in the simulation. 
My work was crucial to make the first fundamental design choices for the novel tracking detectors.

I graduated in June 2014 with distinction. I had the pleasure to be selected to give the presentation 
at the graduation ceremony for all natural sciences.

I started my 3 year post-doc position as COFUND CERN-fellow in fall 2014,  
at the end of the first long shutdown of the LHC in which the LHCb experiment underwent a paradigm shift. 
The detector became the first ever High Energy Physics detector to be aligned, calibrated, 
and fully reconstructed in real-time. Within this novel system, the software trigger decides not only which
events are kept, but also which part of the event information is stored to save bandwidth to offline storage~\cite{LHCb-DP-2016-001}. 
The development of this system was crucial ingredient not only for the Run 2 LHCb physics programme,
but also as a proof-of-principle for the full software trigger of the upgraded LHCb experiment in Run 3. 

My personal role was to develop, configure and commission an online reconstruction
which matches the performance of the previous offline reconstruction. I studied the feasibility 
of the entire concept and led the efforts to convince the LHCb collaboration
of the merits of the approach. This included not only the tracking reconstruction
but also the calorimeter, muon, and Cherenkov detector reconstructions. As part of this work, together with a PhD student,
I deployed a deep neural network to improve the rejection of fake combinations in the pattern recognition,
improving efficiency, fake rate, and timing at the same time.
This is the first application of a deep neural network in the charged particle reconstruction of any LHC experiment. 
I also helped optimize many specific trigger selections such that they exploit the potential of the new reconstruction,
which is crucial in an era where specific particle decays have to 
be triggered and fully selected in real-time. 

Since 2016 I am deputy project leader of the LHCb software trigger, 
responsible for leading an enthusiastic team of several PhD students and post-docs.
My team is primarily in charge of consolidating and developing the Run 2 trigger,
which involves continuously monitoring the trigger performance, validating new trigger
lines when they are added, and continuing to optimize and speed up the reconstruction.
Because of the scale of the problem, involving hundreds of trigger lines which must
operate in parallel without interfering with each other, it is critical to automate
as much as possible the process of designing and testing trigger lines and the reconstruction.
I am particularly focused on this, but I have also continued to make contributions to the 
optimization of the LHCb upgrade reconstruction and the preparations for Run 3, in particular
in the development and testing of the charged particle reconstruction.

I am currently pursuing several analyses, with the objective both of extending our knowledge of fundamental
parameters of the Standard Model and searching for New Physics, and of broadening my own knowledge of physics.
This work has also allowed me to gain experience in supervising students, which I greatly enjoy.
During the restart of the LHC in 2015 at the new centre-of-mass energy I initiated 
the analysis of two-particle correlations produced in proton-proton collisions, 
a topic studied by all experiments at the LHC, giving insight to the mechanisms of hadronization
and collective effects in a dense environment. This summer student project quickly showed 
promising results \cite{Kopecna:2051967}, and the analysis should be published in the coming months.
I am also supervising a masters student in measuring the production asymmetry of $\Lambda_b$ baryons,
which will let us better understand the dynamics of heavy-quark production in 
proton-proton collisions. Again the analysis is mature and approaching publication.

In parallel, and complementary to my previous work on direct $C\!P$-violation measurements
in the charm system, I am collaborating on measuring the charm-mixing and $C\!P$-violation parameters 
with the so-called golden mode $D^0 \rightarrow K_S^0 \pi^+\pi^-$, which is unique in giving direct access to both the
charm mixing parameters (width- and mass-splitting) and the parameters of possible $C\!P$-violation in charm mixing ($|q/p|$ and the 
weak phase $\phi$). The efficiency
to trigger and reconstruct this channel more than doubled due to my work on improving the software trigger.
A crucial difficulty of performing a precise measurement is the knowledge of the Dalitz distribution of the three-body final state,
because it is exactly the interference of the different resonant decay amplitudes which allows the mixing and $C\!P$-violating
observables to be measured. Traditionally this required the modelling of the different resonant contributions, 
which is hard not simply because you have to invent a model which fits the data, but also because of the computational cost of fitting such a multi-dimensional
model to tens of millions of signal events.
Together with a team of postdocs I am currently implementing a novel analysis procedure which allows the measurement to be performed in a model-independent
way, greatly reducing both systematic uncertainties and the computational cost of the measurement.

Finally, I am currently convening one of the subgroups of the top-level Semileptonic working group of LHCb.
As convener, I am responsible for following all analyses carried out within the working group (4 in the last twelve months),
reviewing the internal analysis documentation, and deciding when it is mature enough 
to proceed to the next stage of collaboration review. 
I am particularly focused on using my expertise in the measurement 
of detector efficiencies and asymmetries
to improve the sensitivity of all the physics measurements performed within my group. 
Many individual analyses have developed tools to measure such efficiencies and asymmetries,
and in common with my work on the trigger I am guiding them to make these tools usable by the whole collaboration,
by combining their best features, developing automatic tests and documentation, and making sure that new students
and analysts are aware of these tools and use them.
%One of my goals as a working group convener is to make the tools
%developed by several analyses to measure detection asymmetries more accessible
%and extend their use by colleagues. 

\vspace{2.5cm} 
Geneva, \today

\include{research_proposal3}

\include{list_publications}
%\newpage
%
%\section{Publications with significant personal contributions}
%\cventry{04/2016}{Tesla: an application for real-time data analysis in High Energy Physics}{\href{https://arxiv.org/abs/1604.05596}{arXiv:1604.05596}}{}{}{}
%\cventry{07/2014}{Measurement of $C\!P$ asymmetry in $D^0 \rightarrow K^-K^+$ and $D^0 \rightarrow \pi^- \pi^+$ decays}{\href{http://dx.doi.org/10.1007/JHEP07(2014)041}{DOI: 10.1007/JHEP07(2014)041} }{}{}{}
%\cventry{03/2013}{Search for direct $C\!P$ violation in $D^0 \rightarrow h^-h^+$ modes using semileptonic $B$ decays}{\href{http://dx.doi.org/10.1016/j.physletb.2013.04.061}{DOI: 10.1016/j.physletb.2013.04.061}}{}{}{}
%\cventry{02/2013}{Prompt charm production in $pp$ collisions at $\sqrt{s}$ = 7 TeV}{\href{http://dx.doi.org/10.1016/j.nuclphysb.2013.02.010}{DOI: 10.1016/j.nuclphysb.2013.02.010}}{}{}{}
%\cventry{12/2011}{Measurement of the $C\!P$-violating phase $\phi_s$ in the decay $B^{0}_s \rightarrow J/\psi \phi$}{\href{http://dx.doi.org/10.1103/PhysRevLett.108.101803}{DOI: 10.1103/PhysRevLett.108.101803}}{}{}{}
%\cventry{12/2011}{Measurement of the $B^0_s-\overline{B}^0_s$ oscillation frequency $\Delta m_s$ in $B^0_s \rightarrow D^-_s (3)\pi$ decays}{\href{http://dx.doi.org/10.1016/j.physletb.2012.02.031}{DOI:10.1016/j.physletb.2012.02.031}}{}{}{}
%\cventry{09/2010}{Prompt $K_{S}^{0}$ production in $pp$ collisions at $\sqrt{s}$ = 0.9 TeV}{\href{http://dx.doi.org/10.1016/j.physletb.2010.08.055}{DOI: 10.1016/j.physletb.2010.08.055}}{}{}{}
%

\newpage

%\bibliographystyle{plain}
%\bibliographystyle{plain}
\setboolean{inbibliography}{true}
\bibliographystyle{LHCb}
%\bibliography{publications}
\bibliography{LHCb-PAPER,publications,LHCb-DP}
\end{document}

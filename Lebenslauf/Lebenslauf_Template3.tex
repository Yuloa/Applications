%% start of file `Lebenslauf_Template.tex'.
%% Original-Copyright 2006-2010 Xavier Danaux (xdanaux@gmail.com).
%% Changed by gratiswurst.de
%
%
% This work may be distributed and/or modified under the
% conditions of the LaTeX Project Public License version 1.3c,
% available at http://www.latex-project.org/lppl/.

\documentclass[11pt,a4paper]{moderncv}

% moderncv themes
\moderncvtheme[blue, sans]{casual}                 % optional argument are 'blue' (default), 'orange', 'red', 'green', 'grey' and 'roman' (for roman fonts, instead of sans serif fonts)
%\moderncvtheme[green]{classic}                % idem

% character encoding
\usepackage[english,ngerman]{babel}
\usepackage[ansinew]{inputenc}                   % replace by the encoding you are using

% adjust the page margins
\usepackage[scale=0.8]{geometry}
\usepackage{graphicx}
%\setlength{\hintscolumnwidth}{3cm}						% if you want to change the width of the column with the dates
%\AtBeginDocument{\setlength{\maketitlenamewidth}{6cm}}  % only for the classic theme, if you want to change the width of your name placeholder (to leave more space for your address details
%\AtBeginDocument{\recomputelengths}
\usepackage[T1]{fontenc}
\newcommand{\changefont}[3]{\fontfamily{#1} \fontseries{#2} \fontshape{#3} \selectfont}

\usepackage{setspace}
\usepackage{xcolor}
\definecolor{cred1}{HTML}{CA0614}
\definecolor{cred2}{HTML}{FD3232}


%----------------------------------------------------------------------------------
%            Kontaktdaten
%----------------------------------------------------------------------------------
% VORNAME
\firstname{Claire}
% NACHNAME
\familyname{Prouve}
%FOTO  (optional, ggf. einfach die Zeile l�schen!)
%  64pt = H�he des Bildes, 'picture' = Name des Bildes
%\photo[64pt]{Foto1}

% to show numerical labels in the bibliography; only useful if you make citations in your resumer

%----------------------------------------------------------------------------------
%            Inhalt
%----------------------------------------------------------------------------------
\begin{document}
\maketitle


{\color{cred1} \noindent\rule[3pt]{3.5cm}{2pt} \Large Personal Data}\\
\\
\begin{tabular}{p{3.3cm} p{6cm} p{4cm} l}
 \textbf{Address:} & \quad CERN - Office 13/1-044 & \textbf{Tel. private:} & +41 754117909 \\
 & \quad CH-1211 Geneva 23 & \textbf{E-Mail:} & Claire.Prouve@cern.ch \\
 & \quad Switzerland & &\\
 \textbf{Date of birth:} & \quad 16 March 1987 & \textbf{Nationality:} & German, French \\
\end{tabular}
\\
\newline


{\color{cred2} \noindent\rule[3pt]{3.5cm}{2pt} \Large Education}\\
\\
\begin{tabular}{p{3.3cm} l}
09/2013 - present &{\changefont{pnc}{m}{sc}Ph.D. in Experimental Particle physics} \\
 & \qquad University of Bristol\\
 04/2012 - 06/2013 &{\changefont{pnc}{m}{sc}M.Sc. with specialisation in Experimental Particle physics} \\
 & \qquad RWTH Aachen University  \quad  (grade \textit{very good})\\
  09/2010 - 08/2011 &{\changefont{pnc}{m}{sc}ERASMUS exchange programme} \\
 & \qquad Universit� Paris Sud, participation in the second year master's programme \\
 & \qquad NPAC (Nuclear physics, Particle physics, Astronomy and Cosmology) \\
 10/2006 - 04/2012 & {\changefont{pnc}{m}{sc}B.Sc.in Physics with Computer Science as subsidiary subject} \\
     & \qquad RWTH Aachen University  \\
 09/1997 - 08/2006 &  {\changefont{pnc}{m}{sc}Abitur with specialisation in mathematics and physics}\\
	     & \qquad Gymnasium Horkesgath \\

\end{tabular}
\\
\newline


{\color{cred1} \noindent\rule[3pt]{3.5cm}{2pt} \Large Awards and prizes}\\
\\
\begin{tabular}{p{3.3cm} l}
15/09/2016 & {\changefont{pnc}{m}{sc}LHCb Early-Career Scientist Awards}\\
& \qquad Awarded for the development of the automated RICH mirror alignment  \\
& \qquad within the online data taking framework of LHCb for Run II. This reduced  \\
& \qquad the time the RICH mirror alignment needed to complete from several days \\
& \qquad to 20 minutes and allows to run the alignment every fill instead of once per\\
& \qquad year. This was a significant contribution to the real-time alignment procedure \\
& \qquad and to the understanding of the LHCb RICH detector. \\

19/08/2006 & {\changefont{pnc}{m}{sc}Deutsche Physikalische Gesellschaft Abiturpreis} (book prize) \\ 
& \qquad German physics society prize for pupils with exceptional achievements in \\
& \qquad physics.\\
\end{tabular}
\\
\newline


{\color{cred1} \noindent\rule[3pt]{3.5cm}{2pt} \Large Research Projects}\\
\\
\begin{tabular}{p{3.3cm} l}
 09/2013 - present & {\changefont{pnc}{m}{sc}Ph.D. thesis}\\
 & \qquad \textit{Proper title}\\
 & \qquad \textit{more title} \\
 & \qquad with Dr. J. Rademacker at the University of Bristol  \\
 & \qquad Measurement of the fractional CP-even content $F_+^{4\pi}$ of the self-conjugate decay\\
 & \qquad $D^0 \rightarrow \pi^+\pi^-\pi^+\pi^-$ using Quantum-correlated $\psi(3770)\rightarrow DD$ decays collected\\
 & \qquad by the CLEO-c experiment. The  $D \rightarrow K_S \pi^+\pi^-$ and $D \rightarrow K_L \pi^+\pi^-$ decays were\\
 & \qquad used to tag the signal mode and their binned strong-phase difference used to\\
\end{tabular}
 
\begin{tabular}{p{3.3cm} l}
 & \qquad determine $F_+^{4\pi}$.\\
 & \qquad Ongoing model-independent measurement of the CKM angle $\gamma$ at LHCb\\
 & \qquad using $B^{\pm} \rightarrow D(\rightarrow \pi^+\pi^-\pi^+\pi^-)K^{\pm}$ decays. The $\gamma$ angle is fitted for different\\
 & \qquad  bins in the $ D\rightarrow \pi^+\pi^-\pi^+\pi^-$ phase-space simultaneously. \\
 & \qquad Implementation of the RICH alignment into the real-time alignment framework\\
 & \qquad of LHCb for Run II.\\  
  
 06/2012 - 06/2013 & {\changefont{pnc}{m}{sc}Master's thesis}\\
 & \qquad \textit{Analysis of the $B_d \rightarrow K^{*0} e^+ e^-$ decay at LHCb}\\
 & \qquad with Prof. Dr. M.H. Schune at the Laboratoire de l'Accel�rateur Lin�aire, Paris \\
 & \qquad Analysis of the $B_d \rightarrow K^{*0} e^+ e^-$ decay channel which is particularly sensitive\\
 & \qquad to effects of physics "Beyond the Standard Model" \ by measuring the \\
 & \qquad polarisation of the virtual photon. Evaluating the reconstruction of the \\
 & \qquad soft final state electrons, and optimisation of signal selection procedure using \\
 & \qquad e.g. multivariate analysis tools and determination of signal yields in the \\
 & \qquad 2011 and 2012 LHCb data.\\
 
% & \qquad Study of present limiting factors on the precision measurement and\\
% & \qquad enhancement of the analysis procedure. Examination of the bremsstrahlungs- \\
% & \qquad emission and investigation of different reconstruction- methods \\
% & \qquad and their efficiencies. Optimisation of signal selection procedure using \\
% & \qquad e.g. multivariate analysis tools and determination of signal yields in the \\
% & \qquad Run I data sample recorded by LHCb.\\



 05/2011 - 07/2011 & {\changefont{pnc}{m}{sc}Internship}\\
 & \qquad \textit{Study on the measurement of the CKM angle $\gamma$ at LHCb using untagged} \\
 & \qquad \textit{$B_s \rightarrow D^0 \phi$ events}\\
 & \qquad with Prof. Dr. M.H. Schune at the \textbf{L}aboratoire de l'\textbf{A}ccel�rateur \textbf{L}in�aire, Paris \\
 & \qquad Optimisation of the selection procedure for the $B_s \rightarrow D^0 \phi$ decay using both \\
 & \qquad simulation and data. Selection applied to 135 $pb^{-1}$ 2010/ 2011 data \\
 & \qquad collected by LHCb. No evidence of a signal could be found with this \\
 & \qquad level of statistics. \\
 \end{tabular}

 \begin{tabular}{p{3.3cm} l}
 07/2011 - 09/2011 & {\changefont{pnc}{m}{sc}Bachelor's thesis}\\
 & \qquad \textit{Parametrisation of rigid body dynamics in quaternions } \\ 
 & \qquad with Prof. R. Schmitz at the Institute for theoretical solid state\\
 & \qquad physics (Theory C), RWTH Aachen\\
 & \qquad Study using abstract algebra, quaternion mathematics and rigid body \\
 & \qquad dynamics to evaluate a novel parametrisation of the equations of motion\\
 & \qquad of a rigid body. Deriving the kinematic quantities using quaternions and\\
 & \qquad expressing the Euler equations of rigid body dynamics in quaternion form.\\
 04/2008 - 08/2008 & {\changefont{pnc}{m}{sc}Undergraduate project}\\
 & \qquad with Prof. A. Stahl at the III. Physics Institute B, RWTH Aachen\\
 & \qquad Selected for participation in the radiation therapy research project\\
 & \qquad of the III. Physics Institute B - beyond the standard curriculum.\\
 & \qquad Hardware close programming and calibration of a Charge to Digital\\
 & \qquad Converter (QDC).\\
\end{tabular}
\\
\newline

{\color{cred2} \noindent\rule[3pt]{3.5cm}{2pt} \Large Tasks in the LHCb collaboration}\\
\\
\begin{tabular}{p{3.3cm} l}
05/2015 - present & {\changefont{pnc}{m}{sc}RICH mirror alignment expert} \\
& \qquad Expert on call for the real-time alignment of the RICH mirrors. Training of \\
& \qquad collaboration members. \\
05/2015 - 09/2016 & {\changefont{pnc}{m}{sc}RICH mirror alignment developer}\\
& \qquad Leading the development and implementation of the real-time alignment  \\
& \qquad software for the RICH mirrors.\\
09/2013 - 04/2015 & {\changefont{pnc}{m}{sc}MINT software developer}\\
& \qquad Developer for the only software capable of modelling a generic n-body Dalitz\\
& \qquad plots which is a critical component of the LHCb physics generation software.\\

\end{tabular}
\\
\clearpage

{\color{cred2} \noindent\rule[3pt]{3.5cm}{2pt} \Large Primary Publications}\\
\\
\begin{tabular}{p{3.3cm} l}
forthcoming & {\changefont{pnc}{m}{sc}Model-independent determination of the strong-phase}\\
publication & {\changefont{pnc}{m}{sc}difference between $D^0$ and $\kern 0.2em\overline{\kern -0.2em D^0}{}$ $\to \pi^+\pi^-\pi^+\pi^-$} \\
& \qquad S. Harnew, C. Prouve, J. Rademacker \textit{to be submitted to Physical Review D}\\
forthcoming & {\changefont{pnc}{m}{sc}Amplitude analysis of $D^0 \to \pi^+\pi^-\pi^+\pi^-$ and $D^0 \to K^+K^-\pi^+\pi^-$ decays}\\
publication & {\changefont{pnc}{m}{sc}using CLEO-c data}\\
 & \qquad \href{https://arxiv.org/abs/1611.09253}{P. d'Argent, \textit{et al.} \textit{to be submitted to Physical Review D}} \\
07/2015 & {\changefont{pnc}{m}{sc}First determination of the $C\!P$ content of $D \to \pi^+\pi^-\pi^+\pi^-$ and}\\
& {\changefont{pnc}{m}{sc}updated determination of the $C\!P$ contents of  $D \to \pi^+\pi^-\pi^0$} \\
& {\changefont{pnc}{m}{sc}and  $D \to K^+K^-\pi^0$ } \\
& \qquad \href{http://www.sciencedirect.com/science/article/pii/S0370269315003809}{S. Malde, \textit{et al.} \textit{Phys. Lett. B}, Vol. 747, 01.07.2015, p. 9-17} \\
04/2015 & {\changefont{pnc}{m}{sc}Angular analysis of the $B^0 \rightarrow K^{*0}e^+e^{-} $ decay in the low-$q^2$ region} \\
& \qquad \href{http://link.springer.com/article/10.1007/JHEP04(2015)064}{Aaij, R., \textit{et al.} \textit{J. High Energ. Phys.} (2015) 2015: 64. } \\
05/2013 & {\changefont{pnc}{m}{sc}Measurement of the $B^0 \rightarrow K^{*0}e^+e^-$ branching fraction at}\\
& {\changefont{pnc}{m}{sc}low dilepton mass}\\
& \qquad \href{http://link.springer.com/article/10.1007/JHEP05(2013)159}{Aaij, R., \textit{et al.} \textit{J. High Energ. Phys.} (2013) 2013: 159.} \\
\end{tabular}
\\
\newline



{\color{cred2} \noindent\rule[3pt]{3.5cm}{2pt} \Large Posters and Presentations}\\
\\
\begin{tabular}{p{3.3cm} l}
22/02/2017 & {\changefont{pnc}{m}{sc}Expanding model independent approaches for measuring}\\
& {\changefont{pnc}{m}{sc}the CKM angle $\gamma$}\\
& \qquad Poster at the LHCC \\
14/09/2016 & {\changefont{pnc}{m}{sc}Status of the real-time alignment and calibration activities} \\
& \qquad Plenary talk at the LHCb Week in Santiago de Compostela \\
02/03/2016 & {\changefont{pnc}{m}{sc}Novel Real-time Calibration and Alignment Procedure}\\
&  {\changefont{pnc}{m}{sc}for LHCb Run II} \\
& \qquad Poster at the LHCC \\
14/11/2013 & {\changefont{pnc}{m}{sc}Towards a model-independent measurement of $\gamma$ through} \\
& {\changefont{pnc}{m}{sc}$B^{\pm} \rightarrow D(\rightarrow 4 \pi)K^{\pm}$ decays with LHCb and CLEO-c} \\
& \qquad Poster at the UK High Energy Physics Forum \\
04/04/2013 & {\changefont{pnc}{m}{sc}Photon polarisation in $b\rightarrow s\gamma$ using $B_d \rightarrow eeK^{*}$ at LHCb} \\
& \qquad Poster at the LHC France 2013 conference\\
05/03/2013 & {\changefont{pnc}{m}{sc}Analysis of the rare decay $B_d \rightarrow eeK^{*}$ at LHCb} \\
& \qquad Talk at the 77. Jahrestagung der DPG und DPG-Fr�hjahrstagung\\
\end{tabular}
\\
\newline


{\color{cred1} \noindent\rule[3pt]{3.5cm}{2pt} \Large Additional Training}\\
\\
\begin{tabular}{p{3.3cm} l}
27-30/06/2015 & {\changefont{pnc}{m}{sc}LHCb workshop on multi-body decays of B and D mesons} \\
& \qquad LHCb workshop with invited theorists focussing on the amplitude analysis \\
& \qquad techniques for three- and four-body decays of heavy mesons.\\
 10/2011 - 09/2012 & {\changefont{pnc}{m}{sc}TANDEM Mentoring Programm, RWTH Aachen}\\
 & \qquad Selected for participation in the mentoring programme for talented female\\
 & \qquad students and Ph.D. students. Participation in several trainings aiming\\
 & \qquad at enlarging professional competences and developing key qualifications. \\
 & \qquad  Dr. Tatsuya Nakada as a personal mentor.\\
 7-14/07/2011 & {\changefont{pnc}{m}{sc}Trans-European School of High Energy Physics}\\
18-19/11/2008 & {\changefont{pnc}{m}{sc}Training for Roberta\textsuperscript{\textregistered} workshop -leaders}\\
 & \qquad Acquirement of certificate qualifying me to conduct Roberta\textsuperscript{\textregistered} workshops and \\
 & \qquad trainings Europe-wide.\\
\end{tabular}
\\
\clearpage


{\color{cred1} \noindent\rule[3pt]{3.5cm}{2pt} \Large Teaching Experience}\\
\\
\begin{tabular}{p{3.3cm} l} 
02/2014 - 05/2014 & {\changefont{pnc}{m}{sc}Third year computing course in C}\\ 
& \qquad Tutor, School of Physics, University of Bristol\\
09/2013 - 12/2013 & {\changefont{pnc}{m}{sc} Nuclear and particle physics}\\
& \qquad Teaching assistant, School of Physics, University of Bristol\\
 10/2011 - 03/2012 & {\changefont{pnc}{m}{sc}Programming for everybody - an introduction into JAVA}\\ 
 & \qquad Tutor, 9. Institute for Computer Science, RWTH Aachen\\
 07/2009 - 09/2009 & {\changefont{pnc}{m}{sc}Practical course in physics}\\
& \qquad Laboratory demonstrator, I. Physics Institute B, RWTH Aachen \\
 10/2008 - 03/2009 & {\changefont{pnc}{m}{sc}Theoretical Electrodynamics}\\
 & \qquad Tutor, Institute for theoretical solid state physics (Theory C), RWTH Aachen \\
 04/2008 - 09/2008 & {\changefont{pnc}{m}{sc}Practical course in physics for students with physics}\\
 & {\changefont{pnc}{m}{sc}as subsidiary subject}\\
& \qquad Tutor, I. Physics Institute A, RWTH Aachen\\
 10/2007 - 03/2008 & {\changefont{pnc}{m}{sc}Programming for everybody - an introduction into JAVA}\\
& \qquad Tutor, 9. Institute for Computer Science, RWTH Aachen\\
\end{tabular}
\\
\newline

{\color{cred2} \noindent\rule[3pt]{3.5cm}{2pt} \Large Community Outreach}\\
\\
\begin{tabular}{p{3.3cm} l}
2014 - 2015 &  {\changefont{pnc}{m}{sc}Bristol Bright Night} \\
& \qquad Organisation and execution of interactive and hands-on showcasing of cutting\\
& \qquad edge research for over 1000 people.\\
2013 - 2015 & {\changefont{pnc}{m}{sc}Particle Physics Masterclass}\\
 & \qquad Organisation and execution of several masterclasses in particle physics of the \\
 & \qquad University of Bristol for high-school and A-level students.\\
 2008 - 2013 &  {\changefont{pnc}{m}{sc} Workshop leader for Roberta\textsuperscript{\textregistered} courses}\\
 & \qquad Planning, organisation and execution of workshops to instil young girls into\\
 & \qquad the handling of modern technologies, awake interest in innovative develop-\\
 & \qquad ments in computer science and provide confidence in their technical skills.\\
\end{tabular}  
  
\begin{tabular}{p{3.3cm} l}
 2-4/12/2011 & {\changefont{pnc}{m}{sc} MINT 400 in Berlin}\\
  & \qquad Organisation and realisation of the 5th MINT (Mathematics, Informatics, \\
  & \qquad Natural science, Technology) event, amongst others as exhibition supervisor \\
  & \qquad and workshop leader. The MINT event is a three day seminar for 400 pupils\\
  & \qquad from 147 schools to gain close insight into the MINT working fields and to \\
  &\qquad illustrate the importance of the MINT subjects for our society. \\
 2010/ 2011 & {\changefont{pnc}{m}{sc} Girlsday RWTH Aachen}\\
  & \qquad Organisation and realisation of the annual RWTH Aachen Girlsday.\\
\end{tabular}
\\
\newline

{\color{cred2} \noindent\rule[3pt]{3.5cm}{2pt} \Large Skills}\\
\\
\begin{tabular}{p{3.3cm} l}
 \textbf{Language skills} & \quad Fluent in German, English and French \\
 \textbf{Computing skills} & \quad C++, ROOT, RooFit, python, MINT, JAVA, Geant4 and LHCb software\\
\end{tabular}


\end{document}


%% end of file `Lebenslauf_Template.tex'.
